\documentclass[12pt]{article}
\usepackage{alltt}
\usepackage{fullpage}
\usepackage{multicol,multirow}
\usepackage{tabularx}
\usepackage{ulem}
\usepackage[utf8]{inputenc}
\usepackage[russian]{babel}
\usepackage{indentfirst}

% Оригиналный шаблон: http://k806.ru/dalabs/da-report-template-2012.tex

\begin{document}
	
	\section*{Лабораторная работа №\,3 по курсу дискрeтного \\ анализа: исследование качества программ}
	
	\noindent Выполнила студентка группы 08-208 МАИ \textit{Понагайбо Анастасия}.
	
	\subsection*{Условие}
	
	Для реализации словаря из предыдущей лабораторной работы, необходимо провести исследование скорости выполнения и потребления оперативной памяти. В случае выявления ошибок или явных недочётов, требуется их исправить.
    
	\subsection*{Дневник отладки}
    \begin{center}
 	    \textbf{Valgrind}
    \end{center}
    Была произведена диагностика реализации словаря с помощью Valgrind - инструментального программного обеспечения, предназначенного для отладки использования памяти, обнаружения утечек памяти, а также профилирования.
    Программа была запущена с помощью вызова \textbf{valgrind --log-file=log.txt  --leak-resolution=high  --leak-check=full  --track-origins=yes  ./lab2  <test.txt  >out.txt}. Ключ \textbf{--log-file} задает имя файла, в который будет выводиться отчет о
    работе, ключ \textbf{--leak-resolution=high} сравнивает полный стек вызова функций, \textbf{--leak-check=full} включает функцию обнаружения утечек памяти, \textbf{--track-origins=yes} позволяет отслеживать неинициализированные данные.
    Были выявлены утечки памяти:

	\begin{alltt}
		==4393== Memcheck, a memory error detector
		==4393== Copyright (C) 2002-2013, and GNU GPL'd, by Julian Seward et al.
		==4393== Using Valgrind-3.10.1 and LibVEX; rerun with -h for copyright info
		==4393== Command: ./lab2
		==4393== Parent PID: 4347
		==4393== 
		==4393== 
		==4393== HEAP SUMMARY:
		==4393==     in use at exit: 18,698 bytes in 1,753 blocks
		==4393==   total heap usage: 13,585 allocs, 11,832 frees, 252,658 bytes allocated
		==4393== 
		==4393== 2 bytes in 1 blocks are definitely lost in loss record 1 of 1,019
		==4393==    at 0x402CDAC: operator new[](unsigned int) (in /usr/lib/valgrind/vgpreload_memcheck-x86-linux.so)
		==4393==    by 0x804A465: node::node(char*, unsigned long long) (lab2.cpp:22)
		==4393==    by 0x80491F6: Insert(char*, unsigned long long, node*) (lab2.cpp:121)
		==4393==    by 0x80492D8: Insert(char*, unsigned long long, node*) (lab2.cpp:138)
		==4393==    by 0x80492D8: Insert(char*, unsigned long long, node*) (lab2.cpp:138)
		==4393==    by 0x8049ECB: main (lab2.cpp:372)
		==4393== 
		==4393== 2 bytes in 1 blocks are definitely lost in loss record 4 of 1,019
		==4393==    at 0x402CDAC: operator new[](unsigned int) (in /usr/lib/valgrind/vgpreload_memcheck-x86-linux.so)
		==4393==    by 0x804A465: node::node(char*, unsigned long long) (lab2.cpp:22)
		==4393==    by 0x8049787: Remove(char*, node*) (lab2.cpp:246)
		==4393==    by 0x80495AD: Remove(char*, node*) (lab2.cpp:205)
		==4393==    by 0x80495AD: Remove(char*, node*) (lab2.cpp:205)
		==4393==    by 0x804966C: Remove(char*, node*) (lab2.cpp:220)
		==4393==    by 0x8049F48: main (lab2.cpp:377)
		==4393== 
		==4393== LEAK SUMMARY:
		==4393==    definitely lost: 18,698 bytes in 1,753 blocks
		==4393==    indirectly lost: 0 bytes in 0 blocks
		==4393==      possibly lost: 0 bytes in 0 blocks
		==4393==    still reachable: 0 bytes in 0 blocks
		==4393==         suppressed: 0 bytes in 0 blocks
		==4393== 
		==4393== For counts of detected and suppressed errors, rerun with: -v
		==4393== ERROR SUMMARY: 1019 errors from 1019 contexts (suppressed: 0 from 0)
	\end{alltt}
	Исправив ошибку при удалении, удалось избавиться от утечек:
	\begin{alltt}
		==4468== Memcheck, a memory error detector
		==4468== Copyright (C) 2002-2013, and GNU GPL'd, by Julian Seward et al.
		==4468== Using Valgrind-3.10.1 and LibVEX; rerun with -h for copyright info
		==4468== Command: ./lab2
		==4468== Parent PID: 4347
		==4468== 
		==4468== 
		==4468== HEAP SUMMARY:
		==4468==     in use at exit: 0 bytes in 0 blocks
		==4468==   total heap usage: 13,585 allocs, 13,585 frees, 252,658 bytes allocated
		==4468== 
		==4468== All heap blocks were freed -- no leaks are possible
		==4468== 
		==4468== For counts of detected and suppressed errors, rerun with: -v
		==4468== ERROR SUMMARY: 0 errors from 0 contexts (suppressed: 0 from 0)\\
	\end{alltt}
	 
    \begin{center}
    	\textbf{gprof}
    \end{center}
    
    Предварительно скомпилировав программу с ключом \textbf{-pg} и получив после ее исполнения файл gmon.out, была произведена диагностика с помощью утилиты gprof. Были получены следующие данные:
    
    \begin{alltt}
Each sample counts as 0.01 seconds.
  %   cumulative   self              self     total           
time   seconds   seconds    calls  ms/call  ms/call  name    
56.86     0.58     0.58   624964     0.00     0.00  Insert(char*, unsigned long long, node*)
17.65     0.76     0.18   124967     0.00     0.00  Remove(char*, node*)
15.20     0.92     0.15   250067     0.00     0.00  Find(char*, node*)
3.92      0.95     0.04        1    40.00    40.00  DeleteTree(node*)
2.94      0.98     0.03                             main
0.98      1.00     0.01   285629     0.00     0.00  FixBalance(node*)
0.98      1.01     0.01    46732     0.00     0.00  MinNode(node*, node*)
0.49      1.02     0.01                             Serialise(node*, _IO_FILE*)
0.00      1.02     0.00   620719     0.00     0.00  node::node(char*, unsigned long long)
0.00      1.02     0.00   213076     0.00     0.00  RotateToLeft(node*)
0.00      1.02     0.00   212313     0.00     0.00  RotateToRight(node*)
0.00      1.02     0.00    46732     0.00     0.00  RemoveMin(node*)
    \end{alltt}
    Таким образом, дольше всего работают функции Insert и Remove. Вынеся общие части из блоков в функциях FixBalance, Insert и Remove, удалось немного их ускорить:
    
    \begin{alltt}
Each sample counts as 0.01 seconds.
  %   cumulative   self              self     total           
time   seconds   seconds    calls  ms/call  ms/call  name    
48.05     0.37     0.37   624964     0.00     0.00  Insert(char*, unsigned long long, node*)
18.18     0.51     0.14   250067     0.00     0.00  Find(char*, node*)
15.58     0.63     0.12   124967     0.00     0.00  Remove(char*, node*)
5.19      0.71     0.04                             main
2.60      0.73     0.02    46732     0.00     0.00  MinNode(node*, node*)
1.95      0.75     0.01   620719     0.00     0.00  node::node(char*, unsigned long long)
1.30      0.76     0.01    46732     0.00     0.00  RemoveMin(node*)
1.30      0.77     0.01        1    10.00    10.00  DeleteTree(node*)
0.00      0.77     0.00   285629     0.00     0.00  FixBalance(node*)
0.00      0.77     0.00   213076     0.00     0.00  RotateToLeft(node*)
0.00      0.77     0.00   212313     0.00     0.00  RotateToRight(node*)
    \end{alltt}
	\subsection*{Выводы}
	Утилита Valgrind помогает предотвратить чтение и запись за границами выделенного блока, попытки использования неинициализированной памяти, а также помогает найти утечки памяти, что дает возможность избежать многих ошибок.\\
	\indent Утилита gprof предоставляет информацию о времени работы каждой функции, частоте ее вызова и взаимодействии функций друг с другом, которая может быть в дальнейшем использовано для оптимизации программы.

	
\end{document}